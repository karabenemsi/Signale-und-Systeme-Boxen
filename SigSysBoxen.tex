\documentclass[parskip=half]{scrreprt}
\usepackage[ngerman]{babel}
\usepackage{amsmath}
\usepackage{amsfonts}
\usepackage{amssymb}
\usepackage[utf8x]{inputenc}
\usepackage[T2A,T1]{fontenc}
\usepackage{geometry}
\usepackage{environ}
\usepackage{enumitem}
\usepackage[many]{tcolorbox}
\usepackage{siunitx}


% Title Page
\title{Signale und Systeme Boxen}
\author{Florian Lubitz \& Steffen Hecht}

\newcounter{BoxCounter}
\setcounter{BoxCounter}{0}


\newtcolorbox{dbox}[1][]{%
	enhanced,frame hidden,interior hidden,
	arc=0pt,outer arc=0pt,borderline={0.4pt}{0pt}{dashed},
	nobeforeafter, box align=center, before={\stepcounter{BoxCounter}\boxed{\arabic{BoxCounter}}\makebox[.5cm]{}}, after={\vspace{0.5cm}}
}

\NewEnviron{tbox}{%
	\begin{dbox}
		\BODY
	\end{dbox}
}
\NewEnviron{abox}{%
	\begin{dbox}	
		\begin{align*}
		\BODY
		\end{align*}
	\end{dbox}
}
\makeatletter
\newcommand{\numbereq}{%
	\ifmeasuring@
	\else
	\refstepcounter{equation}%
	\fi
	\tag{\theequation}%
}

% Kyrillische Buchstaben
\newcommand{\KW}{\mbox{\usefont{T2A}{\rmdefault}{m}{n}\CYRSHCH}}


\newcommand*{\rom}[1]{\expandafter\@slowromancap\romannumeral #1@}
\makeatother

\begin{document}
	\maketitle

	
	\chapter{Motivation, Wiederholung und Überblick}
	\begin{tbox}
		a
	\end{tbox}
	\setcounter{BoxCounter}{5}
	

	
	\begin{tbox}
		\begin{align*}
		u_1(t) = \SI{15}{\volt}\sin(\pi t + \pi/3) + \SI{60}{\volt}\sin(10\pi t + \pi/3) = 0,5x(t) + 2y(t)
		\end{align*}
		und damit $a = 0,5, b=2$ und
		\begin{align*}
		u_2(t):= \mathcal{H}\{u_1(t)\} = \mathcal{H}\{0,5x(t) + 2y(t)\} \overset{??}{=}
		\end{align*}
	\end{tbox}
	
	\chapter{Diskrete Signale}
		\setcounter{BoxCounter}{10}
	\begin{tbox}
		\end{tbox}

	\begin{abox}
		(b) \quad x[-k] &= \begin{cases}
			-\frac1k , k /ne 0 \\
			0, k = 0
			\end{cases}
		\\
			(c)\quad  x[k+k_0] &= x[k+3] = \begin{cases}
			\frac{1}{k+3} , k /ne -3 \\
			0, k = -3
			\end{cases}\\
			(d)\quad  x[k-k_0] &= x[k-3] = \begin{cases}
			\frac{1}{k-3} , k /ne 3 \\
			0, k = 3
		\end{cases}	
	\end{abox}

\begin{abox}
 	x[k_0 - k] &= x[-(k - k_0)]\\
 	&= x[(-k) + k_0]
\end{abox}
 
\begin{abox}
 	\text{mit} \quad x[k_0 -k] = x[3 - k] = \begin{cases}
 		\frac{1}{3-k}, k \ne 3\\0 , k = 3
 	\end{cases}
\end{abox}
 
 \begin{tbox}
 	\begin{itemize}
 		\item 

 	$x[k]$ heißt \underline{gerades Signal}, falls $x[k] = x[-k] \forall k \in \mathbb{Z}$ gilt.
 	
 	\item   	$x[k]$ heißt \underline{ungerades Signal}, falls $x[k] = -x[-k] \forall k \in \mathbb{Z}$ gilt.
 	
 	 	\end{itemize}
 	\end{tbox}
 
 \begin{abox}
 	x[-k] = \begin{cases}
 		\frac{1}{-k} , k \ne 0\\
 		0, k = 0
 	\end{cases} = \begin{cases}
 	-\frac1k , k /ne 0 \\ 0, k = 0 
 \end{cases}  = -x [k] 
 \end{abox}
 	
\begin{abox}
 	y[-k] = \begin{cases}
	\frac{1}{(-k)^2} , k \ne 0\\
	0, k = 0
\end{cases} = \begin{cases}
	-\frac{1}{k^2}, k \ne 0 \\
	0, k = 0 
\end{cases}  = y [k] 
 	\end{abox}
 
 
 \begin{tbox}
 	\begin{itemize}
 		\item  $x[k]$ heißt \underline{kausales Signal}, falls gilt: $x[k] = 0 \forall k < 0$
 		\item $x[k]$ heißt \underline{nicht-kausales Signal}, falls gilt $\exists k < 0 : x[k] \ne 0$
 		\item $x[k]$ heißt \underline{anti-kausales Signal}, falls $x[-k-1]$ kausal ist, d.h. falls gilt: $x[k] = 0 \forall k \leqslant 0$
 	\end{itemize}
 \end{tbox}

\begin{tbox}
	\begin{itemize}
		\item $x[k]$ ist nicht-kausal
				\item $u[k]$ ist kausal
						\item $v[k]$ ist anti-kausal
	\end{itemize}
\end{tbox}

\begin{abox}
	\delta[k] := \begin{cases}
		1 , k = 0\\ 0 , k \ne 0
	\end{cases}
\end{abox}

\begin{abox}
	\epsilon[k] := \begin{cases}
		1 , k \ge 0\\ 0 , k < 0
	\end{cases}
\end{abox}

\begin{abox}
	\delta[k-k_0] = \begin{cases}
		1, k = k_0\\ 0, k \ne k_0
	\end{cases}
	\\ \text{bzw.} \\
		\delta[k+k_0] = \begin{cases}
		1, k \ne -k_0\\ 0, k = -k_0
	\end{cases}
\end{abox}

\begin{abox}
	x[k]\cdot \delta[k-i] &= \begin{cases}
		x[i], k = i \\ 0, k \ne i
	\end{cases}\\
	&= x[i] \cdot \delta[k-i] \numbereq \\
\text{Siebeigenschaft}
\end{abox}


\begin{abox}
	x[k] = \sum_{i= -\infty}^{\infty} x[i]\cdot\delta[k-i] \quad \text{für alle} \quad k \in \mathbb{Z}
\end{abox}
	
\begin{abox}
	x[k] = \sum_{i= -K}^{K} x[i]\cdot\delta[k-i]
\end{abox}

\begin{abox}
	u[k] &= \delta[k + 2] + \delta[k + 1] + \delta[k] + \delta[k - 1]\\
	v[k] &= 2\cdot\delta[k + 3] + \delta[k + 1] - \delta[k - 1] - 2\cdot\delta[k - 3]
\end{abox}

\begin{abox}
	sgn[k]:= \epsilon[k] - \epsilon[-k] = \begin{cases}
		1, k > 0\\0, k=0\\-1, k<0
	\end{cases}
\end{abox}

\begin{abox}
	\KW [k] := \epsilon[k] + \epsilon[-k-1] = 1 \text{für alle} k \in \mathbb{Z}
\end{abox}

\begin{abox}
	rect_{k_{1}, k_{2}}[k] :) \epsilon[k-k1] - \epsilon[k-k_2-1] = \begin{cases}
		1, k_1\leqslant k \leqslant k_2
	\end{cases}
\end{abox}

\begin{abox}
	x[k] = q^k \cdot \epsilon[k]
\end{abox}

\begin{abox}
	x[k]: 0, ..., 0, x[0] = 1, x[1]=-0.7,x[2]=0.49,x[3]= 0.343, ...
\end{abox}

\begin{abox}
	x[k]: 0, ..., 0, x[0] = 1, x[1]=-0.8,x[2]=0.64,x[3]= -0.512, ...
\end{abox}

\begin{abox}
	x[k] + y[k] &: x[-\infty] + y[-\infty]  ..., x[0] + y[0], x[1] + y[1], ... , x[\infty] + y[\infty]\\
	x[k] \cdot y[k] &: x[-\infty] \cdot y[-\infty]  ..., x[0] \cdot y[0], x[1] \cdot y[1], ... , x[\infty] \cdot y[\infty]\\
	c \cdot x[k] &: c \cdot x[-\infty]  ..., c \cdot x[0], c \cdot x[1], ... , c \cdot x[\infty]\\
\end{abox}

\begin{abox}
	S_{k_1,k_2} := \{\overrightarrow{x} \in S|x[k] = 0 \forall k < k_1 \text{oder} k>k_2\}
\end{abox}

\begin{abox}
	\overrightarrow{x} &= (0\quad3\quad 2\quad 5\quad 0\quad 0)\\
	\overrightarrow{y} &= (0\quad 0\quad 2\quad -3\quad 0\quad 2)\\
	\overrightarrow{x} + \overrightarrow{y} &= (0\quad 3\quad 4\quad 2\quad 0\quad 2)\\
	\overrightarrow{x} - \overrightarrow{y} &= ( 0\quad 3\quad 0\quad 8\quad 0\quad -2)\\
	\overrightarrow{x} \cdot \overrightarrow{y} &= (0\quad 0\quad 4\quad -15\quad 0\quad 0)\\
		c + \overrightarrow{x} &= ( 0\quad 15\quad 10\quad 25\quad 0\quad 0)
\end{abox}

\begin{abox}
	(x * y)[k] := \sum_{i=-\infty}^{\infty} x[i] \cdot y[k - i]
\end{abox}

\begin{tbox}
	\begin{alignat*}{4}
		& && i = 0 && i = 0 && i = 0\\
		& && \downarrow && \downarrow && \downarrow\\
		& x[i] = &&(3\quad 2\quad 1), \quad y[i] = &&(1 \quad -1 \quad 2)\text{ bzw.}\quad z[0 - i] = (2 \quad -1 \quad && \text{ } 1)
	\end{alignat*}
\end{tbox}

\begin{tbox}
	\begin{tabular}{c | l c c c c c c c | c c}
		& $x[i] =$ & & &3 & 2 & 1 & & & $\sum x[i]y[k - i] =$ & $(x * y)[k]$\\\hline
		$ k = 0$ & $y[k - i] =$ & 2 & -1 & 1 & & & & & $3 \cdot 1$ & $= 3$\\
		$ k = 1$ & & & 2 & -1 & 1 & & & & $3 \cdot (-1) + 2 \cdot 1$ & $=-1$\\
		$ k = 2$ & & & & 2 & -1 & 1 &  & & $3 \cdot 2 + 2 \cdot (-1) + 1 \cdot 1$ & $=5$\\ 
		$ k = 3$ & & & & & 2 & -1 & 1 & & $2 \cdot 2 + 1 \cdot (-1)$ & $=3$\\
		$ k = 4$ & & & & & & 2 & -1 & 1 & $1 \cdot 2$ & $=2$\\
	\end{tabular}
\end{tbox}

\begin{abox}
	x[k] * y[k] = 3 \delta [k] - \delta [k - 1] + 5 \delta [k - 2] + 3 \delta [k - 3] + 2 \delta [k - 4]
\end{abox}

\begin{tbox}
	\begin{alignat*}{2}
		& &&i = -43\\
		& &&\text{ }\downarrow\\
		&x[i]= &&(-1\quad 3\quad -2) \text{ und}\\
		& &&i = 19\\
		& &&\downarrow\\
		& y[i]= &&(1\quad -2\quad 4\quad -1) \text{ bzw. } y[-i] = (-1\quad 4\quad -2\quad 1)
	\end{alignat*}	
\end{tbox}

\begin{tbox}
	$ i = -43$ (pfeil)\\
	\begin{tabular}{c | l c c c c c c c c c | c}
		k & $x[i]$ & & & & -1 & 3 & -2  & & & & $(x * y)[k]$\\\hline
		-24 & $y[k - i] =$ & -1 & 4 & -2 & 1 & & & & & & -1\\
		-23 & & & -1 & 4 & -2 & 1 & & & & & $2 + 3 = 5$\\
		-22 & & & & -1 & 4 & -2 & 1 & & & & $-4-6-2 = -12$\\
		-21 & & & & & -1 & 4 & -2 & 1 & & & $1 + 12 + 4 = 17$\\
		-20 & & & & & & -1 & 4 & -2 & 1 & & $-3 - 8 = -11$\\
		-19 & & & & & & & -1 & 4 & -2 & 1 & 2\\
	\end{tabular}
\end{tbox}

\begin{abox}
	(x * y)[k] = &-\delta [k + 24] + 5\delta [k + 23] - 12\delta [k + 22] + 17\delta [k + 21]\\ &- 11\delta [k + 20] + 2\delta [k + 19]
\end{abox}
 
\begin{abox}
	x[k]*y[k] \in \mathcal{S}_{a+c, b+d} \quad\text{und hat Länge}\quad n+m-1.
\end{abox}

\begin{tbox}
	\begin{enumerate}[label=\Roman*)]
		\item Kommutativität: $x*y = y*x$
		\item Assoziativität: $w*(x*y) = (w*x)*y$ und
								$c\cdot(x*y) = (c \cdot )*y$
		\item Distributivität: $w*(x+y) = w*x + w*y$
		\item Neutrales Element: $x*\delta = x$
		\item Verschiebung: $x[k] * \delta[k_0 - k] = x[k-k_0]$
		\item Zeitinvarianz: $x[k] * y[k-k_0] = (x[k]*y[k])[k-k_0]$
		\item Linearität: $(c\cdot x + d\cdot y)* w = c \cdot (x*w) + d\cdot(y * w)$
	\end{enumerate}
\end{tbox}

\begin{abox}
	p(z) := a_0 + a_1z + a_2z^2 + a_3z^3 + ... + a_nz^n
\end{abox}

\begin{abox}
	x[k] = a_0\delta[k] + a_1\delta[k-1] + a_2\delta[k-2] + ... + a_n\delta[k-n]
\end{abox}

\begin{abox}
	p(z) \cdot q(z) = c_0 + c1_z + c_2z^2 + ... + c_{2n}z^{2n} \quad\text{Mit Koeffizenten } c_k = (c*y)[k]
\end{abox}

\begin{abox}
	p(z) = 3 + 2z + z^2 \text{ und } q(z) = 1 - z + 2z^2
\end{abox}

\begin{abox}
	p(z) \cdot q(z) &= (3+2z +z^2) \cdot (2z^2 - z +1)\\
	&= 3\cdot 1 + z (3\cdot(-1) + 2 \cdot 1) + z^2(3\cdot 2 + 2\cdot (-1)+ 1\cdot1)\\ &\quad + z^3(2\cdot 2 + 1\cdot (-1)) + z^4(1\cdot2)\\
&= 3-z+5z^2 + 3z^3 + 2z^4
\end{abox}

\begin{abox}
E_x := \sum_{i=-\infty}^{\infty} | x[i]|^2
\end{abox}

\begin{abox}
P_x :=  \lim\limits_{K \to \infty}\frac{1}{2K +1}\sum_{i=-K}^{K}|x[i]|^2
\end{abox}

\begin{abox}
\langle x[k],y[k]\rangle_E := \sum_{k=-\infty}^{\infty}x^*[k]\cdot y[k]
\end{abox}

\begin{abox}
	\langle x[k],y[k]\rangle_P := \lim\limits_{K \to \infty} \frac{1}{2K+1} \sum_{k=-K}^{K}x^*[k] \cdot y[k]
\end{abox}

\begin{abox}
	||x[k]||_E := \sqrt{\langle x[k],x[k]\rangle_E} = \sqrt{E_x} \text{ bzw. }\\
		||x[k]||_P := \sqrt{\langle x[k],x[k]\rangle_P} = \sqrt{P_x}
\end{abox}

\begin{abox}
	\cos\Phi = \frac{\langle x[k],y[k]\rangle}{||x[k]|| \cdot ||y[k]||}
\end{abox}

\begin{abox}
	\varphi_{xy}[\kappa]:= \langle x[k], y[k + \kappa] \rangle
\end{abox}

\begin{abox}
	\varphi_{xx}[\kappa]:= \langle x[k], x[k + \kappa] \rangle
\end{abox}

\begin{abox}
	\varphi^E_{xy}[\kappa] = x^*[-\kappa] * y[\kappa] \text{ bzw. } \varphi^P_{xy}[\kappa] = \lim\limits_{K \to \infty} \frac{1}{2K+1} x_K^*[-\kappa] * y_K[\kappa]
\end{abox}

\chapter{Diskrete Systeme}

\setcounter{BoxCounter}{57}
\begin{abox}
	Inhalt...
\end{abox}

\begin{abox}
	y[k] = \mathcal{H}\{x[k]\}
\end{abox}

\begin{tbox}
	\begin{align*}
		x[k] = x_0 \cdot \delta[k] = \begin{cases}
		x_0 , k=0 \\ 0, k\ne 0
		\end{cases}
	\end{align*}
	entwickelt sich nun das Guthaben des Sparbuchs wie folgt:
	
	zu Beginn: $y[0]= x_0$\\
	nach 1 Jahr: $y[1] = x_0 + p \cdot x_0 = (1+p)\cdot x_0$
	nach 2 Jahren: $y[2] = (1+p)x_0 + p\cdot(1+p)\cdot x_0 = (1+p)\cdot(1+p) \cdot x_0 = (1+p)^2 \cdot x_0$\\
	nach 3 Jahren: $y[3] = ... = (1+p)^3 \cdot x_0$\\
	nach $i$ Jahren: $y[i] = (1+p)^i \cdot x_0$
	
	D.h. das Ausgangssignal ist die kausale Exponentialfolge $y[k] = x_0 \cdot (1+p)^k \cdot \epsilon[k]$
\end{tbox}

\begin{tbox}
	\begin{align*}
	y[k+1] = y[k] \cdot (1+p) + x[k+1] \numbereq
	\end{align*}
	Das heißt $y[k+1]$ ergibt sich aus dem verzinsten Guthaben $y[k]$ des vorigen Jahres und zusätzlich den neuen Einzahlungen $x[k+1]$.
\end{tbox}

\begin{abox}
	\mathcal{H}\{c \cdot x_1[k] + d\cdot x_2[k]\} = c \cdot \mathcal{H}\{x_1[k]\} + d \cdot \mathcal{H}\{x_2[k]\}
\end{abox}

\begin{abox}
	y[0] = x[0] = c\cdot x_1[0] + d \cdot x_2[0] 
\end{abox}

\begin{abox}
	y[k+1] &\overset{(3.1)}{=}y[k] \cdot (1+p) + x[k+1]\\
	&\overset{(I.V)}{=} (cy_1[k] + d\cdot y_2[k])\cdot (1+p) + c \cdot x_1 [k+1] + d\cdot x_2[k+1]\\
	&= c\cdot (y_1[k] \cdot (1+p) + x_1[k+1]) + d \cdot (y_2[k]\cdot(1+p) + x_2[k+1])\\
	&\overset{(3.1)}{=} c\cdot y[k+1] + d\cdot y_2[k+1]
\end{abox}

\begin{abox}
	\mathcal{H}\{x[k-k_0]\} = y[k-k_0]
\end{abox}

\begin{abox}
	z[k_0] = x [k_0-k_0] = x[k_0] = y[0] = y[k_0 - k_0]\\
	\text{ und } z[k] = 0 = y[k - k_0] \text{ für } k<k_0
\end{abox}

\begin{abox}
	z[k+1] &\overset{(3.1)}{=} z[k] \cdot (1+p) + x[k+1-k_0]\\
	& \overset{(I.V.)}{=} y[k - k_0] \cdot (1+p) + x[k-k_0 + 1]\\
	& \overset{(3.1)}{=} y[k-k_0 + 1]
\end{abox}

\begin{tbox}
	... wenn der Ausgabewert $y[k_0]$ zur Zeit $k_0$ nur von früheren Eingabewerten $x[k] , k\leq k_0$ abhängig ist.
\end{tbox}

\begin{abox}
	|x[k]| < C \forall k \Rightarrow |y[k]| < D \forall k
\end{abox}

%69
\begin{abox}
	y[k]= x_0 \cdot (1+p)^k \cdot \epsilon[k] \rightarrow \infty \text{ für }k \rightarrow \infty
\end{abox}

\begin{tbox}
	..., wenn der Ausgang $y[k]$ zur Zeit $k$ nur vom Eingang $x[k]$ zur Zeit $k$ abhängt.
\end{tbox}

\begin{tbox}
	..., falls $y[k]$ nur von $x[\kappa]$ für $|\kappa - k| \leq L$ abhängt. 
\end{tbox}

\begin{abox}
	h[k]:= \mathcal{H}\{\delta[k]\}
\end{abox}

\begin{abox}
	y[k] = \mathcal{H}\{x[k]\}
	 &\overset{(2.6)}{=} \mathcal{H}\left\lbrace  \sum_{i = -\infty}^{\infty}x[i] \cdot \delta[k-i] \right\rbrace  \\
&=\sum_{i=-\infty}^{\infty} x[i] \mathcal{H}\{\delta[k-i]\}\\
&=\sum_{i=-\infty}^{\infty} x[i] \cdot h[k-i]\\
&=x[k] * h[k]
\end{abox}

\begin{abox}
	y[k] = x[k] * h[k] \text{ für alle } x[k] \in \mathcal{S}
\end{abox}

\begin{abox}
	h[k] := \mathcal{H}\{\delta[k]\} = (1+p)^k \epsilon[k]
\end{abox}

\begin{abox}
	y[k] &= h[k] * x[k] = \sum_{i=-\infty}^{\infty}(1+p)^i \epsilon[i] \cdot x[k-i]\\
	&= \sum_{i=0}^{\infty}(1+p)^i\cdot x[k-i]
\end{abox}

\begin{abox}
	y[k] &= \sum_{i=0}^{k}(1+p)^i \cdot x[k-i]
\end{abox}

\begin{abox}
	\sum_{y=-\infty}^{\infty} \left|h[i]\right| < \infty
\end{abox}

\begin{abox}
	|y[k]| = |h[k] * x[k]| = |\sum_{i=-\infty}^{\infty}h[k]x[k-i]| \overset{DUG}{\leq} \sum_{i=-\infty}^{\infty}|h[i] \cdot x[k-i]|
	&= \sum_{i=-\infty}^{\infty} |h[i]| \cdot |x[k-i]| < M \sum_{i=-\infty}^{\infty}|h[i]| < M\cdot C < \infty
\end{abox}






\end{document}          
