% !TeX root = ../SigSysBoxen.tex

\begin{abox}
	X(f) &= \sum_{n=0}^{\infty}r_n \cdot \delta(f - nf_0)\\
	&= \hat{x}\frac{T_i}{T} \cdot\delta(f) + \sum_{n=1}^{\infty}2\hat{x}\frac{T_i}{T} \cdot \left|si\left(n\pi\frac{T_i}{T}\right)\right|\cdot \delta(f - nf_0)
\end{abox}

\begin{abox}
	X_H(f) = 2\hat{x}\frac{T_i}{T} \cdot \left|si\left(n\pi\frac{T_i}{T}\right)\right| = 2\hat{x}\frac{T_i}{T} \cdot \left|si\left(\pi\frac{T_i}{T}\cdot\frac{f}{f_0}\right)\right|
\end{abox}

\begin{abox}
	f \in \{4f_0, 8f_0, 12f_0, \dots\}
\end{abox}

\begin{abox}
	c_0 := \frac{a_0}{2} \text{, } c_n := \frac{1}{2}(a_n - jb_n) \text{, }	c_{-n} := \frac{1}{2}(a_n + jb_n) = c_n^*
\end{abox}

\begin{abox}
	a_0 = 2c_0\text{, } a_n = c_n + c_{-n} = 2Re(c_n) \text{, } b_n = j(c_n - c_{-n}) = -2Im(c_n)
\end{abox}

\begin{abox}
	c_n &:= \frac{1}{2}(a_n - jb_n) = \frac{1}{T}\int_{-T/2}^{T/2}x(t)\cos(n\omega_0t)dt - j\frac{1}{T}\int_{-T/2}^{T/2}x(t)\sin(n\omega_0t)dt\\
	&= \frac{1}{T}\int_{-T/2}^{T/2}x(t)(\underbrace{\cos(n\omega_0t) + j\sin(-n\omega_0t)}_{e^{-jn\omega_0t}})dt = \frac{1}{T}\int_{-T/2}^{T/2}x(t) \cdot e^{-jn\omega_0t}dt\\
	c_{-n}& = c_n^* = \frac{1}{T}\int_{-T/2}^{T/2}x(t) \cdot e^{jn\omega_0t}dt
\end{abox}

\begin{abox}
	c_0 &= \frac{a_0}{2} = \hat{x}\frac{T_i}{T}\\
	c_n &= \frac{1}{2}(a_n - jb_n) = \frac{a_n}{2} = \hat{x}\frac{T_i}{T} \cdot \text{si}\left(n\pi\frac{T_i}{T}\right)\\
	c_{-n} &= (c_n)^* = c_n
\end{abox}

\begin{abox}
	c_k = \hat{x}\frac{T_i}{T} \cdot \text{si}\left(k\pi\frac{T_i}{T}\right)
\end{abox}

\begin{abox}
	X_F(t) = \sum_{k = -\infty}^{\infty}c_k \cdot e^{jk\omega_0t} \text{ für } c_k = \frac{1}{T}\int_{-T/2}^{T/2}x(t)e^{-jk\omega_0t}dt
\end{abox}

\begin{abox}
		X_F(t) &= \sum_{k = -\infty}^{\infty}\frac{\varDelta\omega}{2\pi}\int_{-T/2}^{T/2}x(t')e^{-j\omega_kt'}dt' \cdot e^{j\omega_kt}\\
		&= \frac{1}{2\pi} \sum_{k = -\infty}^{\infty}\left[\int_{-T/2}^{T/2}x(t')e^{-j\omega_kt'}dt'\right] \cdot e^{j\omega_kt} \cdot \varDelta\omega
\end{abox}

\begin{abox}
	X_F(t) \overset{(T\to\infty)}{=} \frac{1}{2\pi}\int_{-\infty}^{\infty}\underbrace{\left[\int_{-\infty}^{\infty}x(t) e^{-j\omega t'} \right]}_{=: X(\omega)}  e^{j\omega t} d\omega = \frac{1}{2\pi}\int_{-\infty}^{\infty}X(\omega) e^{j\omega t} d\omega
\end{abox}

\begin{abox}
	x(t) = \delta(t) \slaplace X(f) = \int_{-\infty}^{\infty}\delta(t) \cdot e^{-j2\pi ft}dt = e^0 = 1
\end{abox}

\begin{abox}
	\delta(t) = \frac{1}{2\pi} \int_{-\infty}^{\infty}e^{j\omega t}d\omega = \int_{-\infty}^{\infty}e^{j2\pi ft}df
\end{abox}

\begin{abox}
	x(t) \slaplace X(\omega) &= \int_{-\infty}^{\infty}\hat{x}\cdot\rect(\frac{t}{T_i})e^{-j\omega t}dt = \hat{x}\int_{-T_i/2}^{T_i/2}e^{-j\omega t}dt = \hat{x}\left[\frac{e^{-j\omega t}}{-j\omega}\right]_{-T_i/2}^{T_i/2}\\
	&= -\frac{\hat{x}}{j\omega}\left(e^{-j\omega\frac{T_i}{2}} - e^{j\omega\frac{T_i}{2}}\right) =  -\frac{\hat{x}}{j\omega} \cdot 2j \cdot Im(e^{-j\omega\frac{T_i}{2}})\\
	&= -\frac{2\hat{x}}{\omega}\sin(-\omega\frac{T_i}{2})= \hat{x}T_i\cdot\text{si}(\omega\frac{T_i}{2}) \overset{(\omega = 2\pi f)}{=}\hat{x}\cdot T_i \cdot \text{si}(\pi fT_i) 
\end{abox}

\begin{abox}
	X(f) = \delta(f - f_0) \sLaplace x(t) = \int_{-\infty}^{\infty}\delta(f - f_0)\cdot e^{j2\pi ft}df = e^{j2\pi f_0t}
\end{abox}

\begin{abox}
	c_1x_1(t) + c_2x_2(t) \slaplace c_1X_1(\omega) + c_2X_2(\omega)
\end{abox}

\begin{abox}
	\rect(\frac{t}{2t}) \slaplace 2T\cdot\text{si}(T\omega) \text{ und } \rect(\frac{t}{4t}) \slaplace 4T \cdot \text{si}(2T\omega)
\end{abox}

\begin{abox}
	x(t) &= 2\rect(\frac{t}{2t}) + 0.5\rect(\frac{t}{4t})\\
	&\ztrans\\
	X(\omega) &= 2\cdot 2T\cdot\text{si}(T\omega) + 0.54T \cdot \text{si}(2T\omega)\\
	 &= 4T\text{si}(T\omega) + 2T\text{si}(2T\omega) = 4T\text{si}(\pi 2Tf) + 2T\text{si}(\pi 4Tf)
\end{abox}

\begin{abox}
	\cos(2\pi f_0t) &= \frac{1}{2}e^{j2\pi f_0t} + \frac{1}{2}e^{-j2\pi f_0t} \sLaplace \frac{1}{2}\delta(f - f_0) + \frac{1}{2}\delta(f + f_0)\\
	\sin(2\pi f_0t) &= \frac{1}{2j}e^{j2\pi f_0t} + \frac{1}{2j}e^{-j2\pi f_0t} \sLaplace \frac{1}{2j}\delta(f - f_0) + \frac{1}{2j}\delta(f + f_0)\\
	&= \frac{j}{2}\delta(f + f_0) - \frac{j}{2}\delta(f - f_0)
\end{abox}

\begin{abox}
	x(t - t_0) \slaplace e^{-j\omega t_0}X(\omega)
\end{abox}

\begin{abox}
	x(t) = \rect(\frac{t - t_0}{T})\slaplace X(\omega) = e^{-j\omega_0t} \cdot T \cdot \text{si}(\pi fT)
\end{abox}

\begin{abox}
	x(at) \slaplace \frac{1}{|a|}X(\frac{\omega}{a})\\
	\text{ insbesondere also } x(-t) \slaplace X(-\omega)
\end{abox}

\begin{abox}
	x_2(t) = \rect(0,5t) \slaplace X_2(f):=2\text{si}(\pi \frac{f}{0,5}) \text{ und }\\
	x_3(t) = rect(2t) \slaplace X_3(f):= \frac{1}{2} \text{si}(\pi \frac{f}{2})
\end{abox}

\begin{abox}
	\delta(at) \slaplace \frac{1}{|a|} \sLaplace \frac{1}{|a|} \delta(t) \text{, d.h. } \delta(at) = \frac{1}{|a|}\delta(t)
\end{abox}

\begin{tbox}
	\begin{enumerate}[label=\Roman*)]
		\item Falls $x(t) \slaplace X_{\omega}(\omega)$ gilt, dann gilt auch $X_{\omega}(t) \slaplace 2\pi x(-\omega)$
		\item Falls $x(t) \slaplace X_{f}(f)$ gilt, dann gilt auch $X_{f}(t) \slaplace x(-f)$
	\end{enumerate}
\end{tbox}


\begin{abox}
	1 \slaplace 2\pi \delta(-\omega) = 2\pi \delta(-\omega) \text{ bzw. } 1 \slaplace \delta(-f) = \delta(f)
\end{abox}

\begin{abox}
	X(t) &= T' \text{si}(\pi T't) \slaplace \rect(\frac{-f}{T'}) = rect(\frac{f}{T'})\\
	&\overset{(T' = \frac{1}{T})}{\Longleftrightarrow} \frac{1}{T}\text{si}(\pi\frac{t}{T}) \slaplace \rect(T\cdot f)\\
	&\overset{(lin.)}{\Longleftrightarrow} \text{si}(\pi \frac{t}{T}) \slaplace T \rect(Tf)		\numbereq
\end{abox}

\begin{abox}
	\frac{d}{dt}x(t) \slaplace j\omega X(\omega)
\end{abox}

\begin{abox}
	x''(t)+3x'(t)+x(t) &= \rect(t)\\
	&\ztrans\\
	(j\omega)^2X(\omega)+3j\omega X(\omega) + X(\omega)&=\text{si}(\frac{\omega}{2})
\end{abox}

\begin{abox}
	X(\omega) = \frac{\text{si}(\frac{\omega}{2})}{(j\omega)^2 + 3j\omega + 1}
\end{abox}

\begin{abox}
	\text{sgn}'(t) &:= \epsilon'(t) - (\epsilon(-t))' = \delta(t) - \delta(-t) \cdot (-1) = 2\delta(t)\\
	&\ztrans\\
	j\omega Y(\omega) &= 2 \cdot 1 \Longleftrightarrow Y(\omega) = \frac{2}{j\omega} \overset{(\omega = 2\pi f)}{=} \frac{1}{j\pi f}
\end{abox}

\begin{abox}
	\epsilon(t) &= \frac{1}{2} + \frac12\text{sgn}(t)\\
	&\ztrans\\
	X(f) &= \frac12 \delta(f) + \frac{1}{j2\pi f} \overset{(5.29)}{=} \pi \delta(\omega) + \frac{1}{j\omega}
\end{abox}

\begin{abox}
	e^{at} \cdot \epsilon(t) \slaplace &\int_{-\infty}^{\infty}e^{at}\epsilon(t) \cdot e^{-j\omega t}dt = \int_{0}^{\infty}e^{(a-j\omega)t}dt\\
	&= \left[\frac{e^{(a-j\omega)t}}{a-j\omega}\right]_{0}^{\infty} \overset{(a<0)}{=} 0-\frac{1}{a-j\omega}\\
	&= \frac{1}{j\omega-a} = \frac{1}{j2\pi f -a}
\end{abox}


\begin{tbox}
	\begin{enumerate}[label=\Roman*)]
		\item Faltungstheorem: $x(t) * y(t) \slaplace X(\omega) \cdot (\omega)$
		
		\item Multiplikationstheorem: $x(t) \cdot y(t) \slaplace \frac{1}{2\pi}X(\omega) * Y(\omega)$ (bzw. \slaplace $X_f(f) * Y_f(f)$)
	\end{enumerate}
\end{tbox}

\begin{abox}
	Y(f) = X(f) \cdot \rect(\frac{f}{2f_g}) = \begin{cases}
		X(f)&,|f| < f_g\\
		0 &,|f| > f_g
	\end{cases}
\end{abox}

\begin{abox}
	Y(f) &= X(f) \cdot \rect(\frac{f}{2f_g})\\
	& \ztransrueck\\
	y(t) &= x(t) * 2f_g \cdot \si(s\pi f_gt)
\end{abox}

\begin{abox}
	\int_{-\infty}^{t}x(\tau)d\tau \slaplace \frac{1}{j\omega}X(\omega) + \pi \cdot X(0) \cdot \delta(\omega)\\
	= \frac{1}{j2\pi f} X_f(f) + \frac12 X_f(0) \cdot \delta(f)
\end{abox}

\begin{abox}
	e^{j\omega t}\cdot x(t) \slaplace X(\omega - \omega_0)\quad(=X_f(f-f_0))
\end{abox}

\begin{abox}
	X(j\omega) = \int_{-\infty}^{\infty}e^{5t}\epsilon(t)\cdot e^{-j\omega t} dt = \int_{0}^{\infty}e^{(5-j\omega)\cdot t} dt = \left[\frac{e^{(5-j\omega)\cdot t}}{5-j\omega}\right]_0^\infty \to \infty
\end{abox}

\begin{abox}
	x(t) \cdot e^{-\sigma t} = e^{5t} \cdot \epsilon(t) \cdot e^{-6t} = e^{-t} \cdot \epsilon(t)
\end{abox}

\begin{abox}
	\mathcal{L}\left\lbrace x(t)\right\rbrace  := \mathcal{F}\left\lbrace x(t)\cdot \epsilon^{-\sigma t}\right\rbrace 
\end{abox}

\begin{abox}
	\mathcal{L}\left\lbrace x(t) \right\rbrace &= \int_{-\infty}^{\infty} x(t) e^{-\sigma t} \cdot e^{j\omega t} dt = \int_{-\infty}^{\infty} x(t) e^{-(\sigma + j\omega) t} dt\\
	&= \int_{-\infty}^{\infty} x(t) \cdot e^{-st}dt =: X_{\mathcal{L}}(s)
\end{abox}

\begin{abox}
	x(t) e^{-dt} = \mathcal{F}^{-1}\left\lbrace X_{\mathcal{L}}(s) \right\rbrace = \frac{1}{2\pi} \int_{-\infty}^{\infty} X_{\mathcal{L}}(\sigma+j\omega)\cdot e^{j\omega t}d\omega
\end{abox}

\begin{abox}
	x(t) = \frac{1}{2\pi}\int_{-\infty}^{\infty}X_{\mathcal{L}}(\sigma + j\omega) \cdot e^{(\sigma + j\omega)\cdot t}d\omega = \frac{1}{2\pi j} \int_{\sigma -j\infty}^{\sigma + j \infty}X_{\mathcal{L}}(s)\cdot e^{st} ds
\end{abox}

\begin{abox}
	\delta(t)\slaplace \int_{-\infty}^{\infty}\delta(t)\cdot e^{-st}dt = e^{-s\cdot 0} = 1 \text{  für  } s \in \mathbb{C}
\end{abox}

\begin{abox}
	e^{at}\epsilon(t) \slaplace \int_{-\infty}^{\infty}e^{at}\epsilon(t)e^{-st}dt = \int_{0}^{\infty}e^{(a-s)t}dt
	\\= \left[\frac{e^{(a-s)t}}{a-s}\right]_{0}^{\infty} \overset{\text{Re}(a-s)<0}{=} e^{-\infty} - \frac{e^{0}}{a-s} = \frac{1}{s-a} \text{\quad für \quad } \text{Re}(s) > \text{Re}(a)
\end{abox}

\begin{abox}
	x(t)&=\cos(\omega_0 t) \cdot \epsilon(t) = \frac{1}{2}e^{j\omega_0 t}\epsilon(t) + \frac{1}{2}e^{-j\omega_0 t}\epsilon(t)\\
	&\ztrans\\
	X(s) &= \frac12 \frac{1}{s-j\omega_0} + \frac12 \frac{1}{s+j\omega_0} = \frac12 \frac{s-j\omega_0 + s+j\omega_0}{(s+j\omega_0)(s-j\omega_0)}\\
	&= \frac{s}{s^2 + \omega_0^2} \qquad \text{für} \quad \text{Re}(s)>0
\end{abox}

\begin{abox}
	&u(t) &= R \cdot i(t) \quad\text{bzw}\quad &u(t) &= L \cdot \frac{d}{dt} i(t) \quad \quad\text{bzw}\quad &i(t)&= C \cdot \frac{d}{dt}u(t)\\
	 & &\ztrans & &\ztrans & &\ztrans\\
	 &U(s) &= R \cdot I(s)  &U(s) &=L \cdot sI(s)  &I(s)&=C\cdot s \cdot U(s)
\end{abox}

\begin{abox}
	H_R(s):=\frac{U(s)}{I(s)} = R \quad\text{bzw}\quad H_L(s):= \frac{U(S)}{I(S)} = sL \quad\text{bzw}\quad H_C(s) := \frac{U(s)}{I(s)} = \frac{1}{sC}
\end{abox}


\begin{abox}
	\text{MR}: &u_1(t) + u_2(t) + ... + u_n(t) = 0\\
	&\ztrans\\
	&U_1(t) + U_2(t) + ... + U_n(t) = 0\\	
	\text{KR}: &i_1(t) + i_2(t) + ... + i_n(t) = 0\\
	&\ztrans\\
	&I_1(t) + I_2(t) + ... + I_n(t) = 0
\end{abox}

\begin{tbox}
	a) $u_1(t) = 10\epsilon(t) \laplace U_1(s) = \frac{10}{s}$\\
	b) $u_1(t) = 3\delta(t) \laplace U_1(s) = 3$\\
	c) $u_1(t) = 2e^{4t}\cdot \epsilon(t) \laplace U_1(s) = \frac{2}{s-4}$
	
	$H_R(s) = 2000=R, H_C(s) = \frac{1}{sC} = \frac{1000}{s}$
\end{tbox}


\begin{abox}
	H(s):= \frac{U_2(s)}{U_1(s)} = \frac{H_R(s)}{H_R(s)+H_C(s)} = \frac{R}{R+\frac{1}{sC}} = \frac{RCs}{RCs + 1} = \frac{2s}{2s+1}
\end{abox}

\begin{tbox}
	a) $U_2(s) = H(s) \cdot U_1(s) = \frac{2s}{2s+1}\cdot \frac{10}{s} = \frac{20}{2s+1}$\\
	b) 	$U_2(s) = H(s) \cdot U_1(s) = \frac{2s}{2s+1}\cdot 3 = \frac{6s}{2s+1}$\\
	c)		$U_2(s) = H(s) \cdot U_1(s) = \frac{2s}{2s+1}\cdot \frac{2}{s-4} = \frac{4-s}{(2s+1)(s-4)}$
\end{tbox}

\begin{tbox}
	a) $U_2(s) = \frac{20}{2s + 1}= 10\frac{1}{s+0,5} \sLaplace u_2(t) = 10e^{-0,5t}\epsilon(t)$\\
	b) $U_2(s) = \frac{6s}{2s + 1} = 3s\frac{1}{s+0,5} \sLaplace  3 \cdot \frac{d}{dt} \left[ e^{-0,5t}\epsilon(t)\right] $\\
	$\quad\implies u_2(t) = 3 \cdot( e^{-0,5t}\cdot -0,5 \cdot \epsilon(t) + e^{-0,5t} \cdot \delta(t)) = 3\delta(t) - 1,5e^{-0,5t}\epsilon(t) $
	c) $U_2(s) \sLaplace u_2(t) = ...$ s.h. Übungen mit PBZ
\end{tbox}

\chapter{Analoge Signale}
\begin{abox}
	Inhalt...
\end{abox}


\begin{abox}
	\mathcal{H}\left\lbrace c \cdot x_1(t) + d\cdot x_2(t)\right\rbrace = c \cdot \mathcal{H}\left\lbrace x_1(t) \right\rbrace + d \cdot \mathcal{H}\left\lbrace x_2(t) \right\rbrace
\end{abox}

\begin{abox}
	y(t-t_0) = \mathcal{H}\left\lbrace x(t-t_0) \right\rbrace
\end{abox}

\begin{abox}
	y(t) = x(t) * h(t) \quad \text{für alle } x(t) \in \mathcal{V}
\end{abox}

\begin{abox}
	Y(s) = X(s) \cdot H(s)  \quad \text{für alle } x(t) \in \mathcal{V}
\end{abox}

\begin{abox}
	H(s) = \frac{Y(s)}{X(s)}
\end{abox}

\begin{abox}
	y(t) &= x(t) * h(t) \quad \text{für alle } x(t) \in \mathcal{V}\\
	&\ztrans\\
	Y(s) &= X(s) \cdot H(s) \quad \text{für alle } X(s) \slaplace x(t) \in \mathcal{V}
\end{abox}

%261
\begin{abox}
	y(t) &= \mathcal{H}\left\lbrace x(t) \right\rbrace = \mathcal{H}\left\lbrace\int_{-\infty}^{\infty}x(\tau)\cdot \delta(t-\tau)d\tau\right\rbrace &\text{(wegen Satz 4.4.III)}\\
	&= \int_{-\infty}^{\infty}x(\tau)\cdot{H}\left\lbrace\delta(t-\tau)d\tau\right\rbrace &\text{(Linarität)}\\
	&= \int_{-\infty}^{\infty}x(\tau)\cdot h(t-\tau) d\tau\\
	&= x(t) * h(t) &\text{(Def. Faltung, s.h. Seite 89)}
\end{abox}

\begin{abox}
	H(s) = \frac{RCs}{RCs + 1} &= s\cdot \frac{1}{s + \frac{1}{RC}}\\
	&\ztrans\\
	h(t) &= \frac{d}{dt}\left[e^{-\frac{t}{RC}}\cdot\epsilon(t)\right] = e^{-\frac{t}{RC}} \cdot -\frac{1}{RC}\cdot \epsilon(t) + e^{-\frac{t}{RC}}\cdot\delta(t)\\
	&= \delta(t) -\frac{1}{RC} \cdot e^{-\frac{t}{RC}}\epsilon(t) = \delta(t) - \frac{1}{2}e^{-\frac{t}{2}}\epsilon(t)
\end{abox}

\begin{abox}
	y(t) &= h(t) * x(t) = (\delta(t) -\frac{1}{RC} \cdot e^{-\frac{t}{RC}}\epsilon(t)) * x(t)\\
	&= x(t) - -\frac{1}{RC} \cdot \int_{-\infty}^{\infty} e^{-\frac{\tau}{RC}}\epsilon(\tau) \cdot x(t-\tau) d\tau\\
	&=  x(t) - -\frac{1}{RC} \cdot \int_{0}^{\infty} e^{-\frac{\tau}{RC}}\epsilon(\tau) \cdot x(t-\tau) d\tau\\
	&=  x(t) - -\frac12 \cdot \int_{0}^{\infty} e^{-\frac{\tau}{2}}\epsilon(\tau) \cdot x(t-\tau) d\tau
\end{abox}


\begin{abox}
	y(t) &= \epsilon(t) - \frac{1}{RC} \int_{0}^{\infty} e^{-\frac{\tau}{RC}}\cdot \epsilon(t-\tau) d \tau\\
	&= \epsilon(t) - \frac{1}{RC} \int_{0}^{\text{max}(t,0)} e^{-\frac{\tau}{RC}}d\tau\\
	&= \epsilon(t) - \frac{1}{RC}\left[\frac{e^{-\frac{\tau}{RC}}}{-\frac{1}{RC}} \right]^{\text{max}(t,0)} = \epsilon(t) + \left(e^{-\frac{t}{RC}} -1 \right) \cdot \epsilon(t) = e^{-\frac{t}{RC}}\epsilon(t) = e^{-\frac{t}{2}}\epsilon(t)
\end{abox}

\begin{abox}
	i_R(t) = \frac1R \cdot u_R(t) \quad \text{btw.} \quad i_C(t) = C\cdot\frac{d}{dt}u_C(t)
\end{abox}

\begin{abox}
	u_1(t) = u_R(t) + u_C(t)
\end{abox}

\begin{abox}
	x(t) = RC\frac{d}{dt}y(t) + y(t)
\end{abox}

\begin{abox}
	x(t) &= RC \frac{d}{dt} y(t) + y(t)\\
	&\ztrans\\
	X(s) &= RC\cdot s \cdot Y(s) + Y(s)
\end{abox}

\begin{abox}
	X(s) &= RC\cdot s \cdot Y(s) + Y(s) \Leftrightarrow X(s) = Y(s) \cdot (RCs + 1)\\
	&\implies H(s) = \frac{Y(s)}{X(s)} = \frac{1}{RCs + 1}
\end{abox}

\begin{abox}
	Y(s) = H(s) \cdot X(s) = \frac{1}{RCs + 1} \cdot \frac1s = \frac{1}{(RCs + 1) s} = \frac{1}{RC} \cdot \frac{1}{(s+ \frac{1}{RC})\cdot s}
\end{abox}

\begin{abox}
	Y(s) = \frac{1}{RC} \cdot( \frac{A}{s-\alpha} + \frac{B}{s-\beta}) =  \frac{1}{RC} \cdot( \frac{-RC}{s-\frac{1}{RC}} + \frac{RC}{s}) = \frac1s - \frac{1}{s+\frac{1}{RC}}
\end{abox}

\begin{abox}
	Y(s) &= \frac1s - \frac{1}{s+\frac{1}{RC}}\\
	&\ztransrueck\\
	y(t) = \epsilon(t) - e^{-\frac{t}{RC}}\cdot \epsilon(t) = (a-e^{-\frac{t}{RC}})\cdot \epsilon(t)
\end{abox}